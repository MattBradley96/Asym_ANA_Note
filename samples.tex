\section{Data and simulated samples} \label{sec:samples}

\subsection{Data samples}
The analysis is performed using the full Run-II dataset collected by \lhcb at 13\tev, corresponding to a total integrated luminosity of 5.9\invfb. 

\textit{Will be lots to say here on different stripping lines etc. I'm sure - could take from cross section analysis but thought I'd wait until we actually have the full data sample.
}
\subsection{Simulation samples}
For the analysis, simulation samples of \bbbar, \ccbar and \qqbar dijets (where q stands for a light parton $u$, $d$, $s$ or $g$) were required for a few purposes:
\begin{itemize}
    \item to model the distributions of SV-tagging related observables for the different flavours. These are used in the fit for measuring sample purity,
    \item to produce response matrices that can be used to unfold detector effects from the measured distributions,
    \item to estimate the performance of the qTAG procedure.
\end{itemize}
For each dijet type a simulation sample was produced with and without the condition that a muon be present in the final \lhcb acceptance. This allowed for the efficiency of qTAG to be calculated, whilst also ensuring there was a significant number of events with muons present to analyse the performance further. The six event types were produced in three bins of $m_{bb}$ to ensure there was a significant amount of events across the full range of the measurement.